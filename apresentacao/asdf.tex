\documentclass[12pt]{beamer}

\usepackage[portuges]{babel}
\usepackage{lmodern}
\usepackage[utf8]{inputenc}
\usepackage{amssymb,amsmath}
\usepackage{amssymb}
\usepackage[T1]{fontenc}
\usepackage{graphicx}
\usepackage{caption}
\usepackage{verbatim}

\usepackage{bold-extra}
\usepackage{minted}
\setbeamertemplate{footline}[frame number]

%===== Pacotes para colorir links =====
\usepackage{fancyhdr}
%\usepackage[colorlinks,linkcolor=blue,hyperindex]{hyperref}
%\hypersetup{backref,  pdfpagemode=FullScreen, colorlinks=true,linkcolor=blue}

%===== Pacotes para mostrar códigos fontes =====
\usepackage{listings}

% Caminhos das Imagens
\graphicspath{{images/}}

%===== Configurações para mostrar Códigos Fonte ===== %
\lstset{numbers=left,
  language=python,
  stepnumber=1,
  firstnumber=1,
  numberstyle=\tiny,
  extendedchars=false,
  escapeinside='',
  breaklines=true,
  frame=tb,
  basicstyle=\tiny,
  stringstyle=\ttfamily,
  showstringspaces=false
  backgroundcolor=\icolor{gray}
  morecomment=[l]{//} % displays comments in italics (language dependent)
}

\author[Marczal]{Diego Marczal\\Josiel Neumann Kuk}
\title{Introdu{\c c}\~ao ao \LaTeX}
\institute{IV Jornada de Atualiza{\c c}\~ao em Inform\'atica \\ Minicurso de \LaTeX}
\date{2011}

%\mode<beamer>{ \usetheme{Frankfurt} } %Ilmenau
 \usetheme{Frankfurt}
\begin{document}

%\input{slides/capa}
%%-------------------------------------------------------------
%-------------------------------------------------------------

%================ Slide Sumario===============================
% cria o sumário
\begin{frame}
\frametitle{Sumário}
\tableofcontents[pausesections]
\end{frame}
%------------------------------------------------------------
%============================================================

%\section{Hist\'orico}
\subsection{O que \'e \LaTeX?}


%----------------O que é Tex -------------------
\begin{frame}
\frametitle{O que \'e \TeX?}

\begin{itemize}

\item Criado originalmente por Donald E. Knuth (\TeX em 1977);
\item Desenvolvido para escrever livros com alta qualidade;
\item Knuth afirma que o \TeX n\~ao tem \textit{bugs};
\item O n\'umero da vers\~ao converge para $\pi$;
\item Pron\'uncia correta \'e "Tech" (no entanto existe a variante "Teks").

\end{itemize}

\end{frame}


%----------------O que é LaTex -------------------
\begin{frame}
\frametitle{O que \'e \LaTeX}

\begin{itemize}

\item Conjunto de macros que permitem a cria{\c c}\~ao de documentos com leioute pr\'e-definido;
\item Desenvolvido por Leslie Lamport;
\item O \LaTeX \'e um programa de c\'odigo aberto;
\item Existem v\'arias implementa{\c c}\~oes (TeTex, TexLive, MikTeX, etc);
\item A pron\'uncia correta \'e "Lay-Tech". Existem variantes como "Lah-Tech" e "Lah-Teks".

\end{itemize}

\end{frame}

%---------------- Vantagens -------------------
\begin{frame}
\frametitle{Vantagens}

\begin{itemize}

\item Resultado superior (melhor formata{\c c}~ao e qualidade tipogr\'afica);
\item Portabilidade, estabilidade e disponibilidade;
\item Seus documentos podem ser facilmente e corretamente estruturados;
\item \'Indices, notas de rodap\'e e referencias s\~ao geradas facilmente;
\item F\'ormulas matem\'aticas tamb\'em s\~ao facilmente criadas;

\end{itemize}
\end{frame}

%---------------- Desvantagens -------------------
\begin{frame}
\frametitle{Desvantagens}

\begin{itemize}

\item Uso de ferramentas auxiliares;
\item Trabalha-se diretamente com o "c\'odigo" e n\~ao o "visual" (com exce{\c c}\~oes);
\item \'E necess\'ario ter conhecimento dos comandos \LaTeX;
\item Em alguns momentos pode ser dif\'icil de conseguir alguns "\textit{looks}".

\end{itemize}
\end{frame}

%\section{Instala{\c c}\~ao}

\subsection{Instala{\c c}\~ao em Windows e Linux}
%----------------------------------------------------------------------------%

\begin{frame}
 
\begin{block}{Instala{\c c}\~ao	Windows}
Tr\^es itens obrigatórios (nessa ordem):
\begin{itemize}
\item GhostScript;
\item GhostView;
\item MiK\TeX.
\end{itemize}
Item opcional:
\begin{itemize}
\item Algum editor.
\end{itemize}
\end{block}

\begin{block}{Instala{\c c}\~ao Linux}
Instalar uma das distribui{\c c}\~oes abaixo:
\begin{itemize}
\item TeX-Live (recomendado);
\item TeTex.
\end{itemize}
Item opcional:
\begin{itemize}
\item Algum editor.
\end{itemize}
\end{block}

\end{frame}




%\section{Conceitos B\'asicos}
\subsection{Estrutura B\'asica}

\begin{frame}{Arquivo fonte}

\begin{itemize}
\item Um arquivo fonte cont\'em o texto a ser processado e comandos que indicam como o texto deve ser processado;
\item Palavras s\~ao separadas por um ou mais espa{\c c}os;
\item Par\'agrafos s\~ao separados por uma ou mais linhas;
\item O resultado final n\~ao \'e alterado pela quantidade de espa{\c c}os/linhas;
\item A maioria dos comandos come{\c c}am com o caracter \;
\item O texto n\~ao deve conter caracteres especiais diretamente.

\end{itemize}
\end{frame}




\begin{frame}[fragile]{Estrutura b\'asica de um documento}
\begin{block}{Modelo b\'asico}
\verb| \documentclass[opções]{class/estilo} |\\
\verb| %preambulo - inclusão de pacotes, etc|\\
\verb| \usepackage[opções]{pacote}|\\
\verb| \begin{document}|\\
\verb| %conteúdo do documento|\\
\verb| \end{document}|
\end{block}
\end{frame}


\begin{frame}[fragile]{Elementos b\'asicos de um documento}

Abaixo est\~ao descritos algumas classes de documentos que podem ser utilizadas com \LaTeX:
\begin{description}
\item[article] Para artigos em revistas cient\'ificas, apresenta{\c c}\~oes, pequenos relat\'orios, convites, etc;
\item[proc] Uma classe para \textit{proceedings} baseada na classe de artigos;
\item[minimal] Essa classe gera um documento m\'inimo (apenas define um tamanho de p\'agina e uma fonte base);
\item[report] Para longos relat\'orios contendo muitos cap\'itulos, pequenos livros, teses, etc;
\item[book] Para cria{\c c}\~ao de livros;
\item[slides] Classe para cria{\c c}\~ao de \textit{slides};
\item[memoir] \'E baseado na classe para cria{\c c}\~ao de livros, com altera{\c c}\~oes na produ{\c c}\~ao do documento;
\item[letter] Para gera{\c c}\~ao de cartas. 
\end{description}
\end{frame}

\begin{frame}[fragile]{Elementos b\'asicos de um documento}

Abaixo est\~ao listadas as op{\c c}\~oes mais comuns para os documentos de classes padr\~oes \LaTeX:
\begin{description}
\item[10pt, 11pt, 12pt] Define o tamanho da fonte principal do documento;
\item[a4paper, letterpaper,...] Define o tamanho do papel. O tamanho padr\~ao \'e \texttt{letterpaper};
\item[fleqn] Apresenta as f\'ormulas alinhadas \`a esquerda ao inv\'es de centradas;
\item[titlepage, notitlepage] Especifica se uma nova p\'agina deve ser iniciada ap\'os o t\'itulo do documento ou n\~ao;
\item[onecolumn,twocolumn] Define se o texto estar\'a em uma ou duas colunas;
\item[landscape] Altera o leioute do documento para impress\~ao em modo paisagem;
\end{description}
\end{frame}


\begin{frame}[fragile]{Comandos}
\begin{itemize}
\item Um comando em \LaTeX  \'e normalmente precedido de \backslash e seguido de par\^ametros opcionais (delimitados por $[$ e $]$) e/ou par\^ametros obrigat\'orios (delimitados por ${$ e $}$);
\item Uma exce{\c c}\~ao a esta regra \'e o cifr\~ao ("\$"), que delimita o ambiente matemático;
\begin{itemize}
\item Exemplo: \verb|$ax^2+bx+c=0$| produz $ax^2+bx+c=0$.
\end{itemize} 
\end{itemize}

\end{frame}

%\section{Funcionamento}
\subsection{Exemplos b\'asicos}

\begin{frame}[fragile]{Exemplo b\'asico}
\verb| % Este é um pequeno arquivo fonte para o LaTeX| \\
\verb| % O símbolo "%" indica um comentário e é ignorado| \\ 
\verb| \documentclass[10pt]{article}| \\
\verb| \usepackage[latin1]{inputenc}| \\
\verb| \usepackage[brazil]{babel}| \\
\verb| \usepackage{graphicx}| \\
\verb| \begin{document}| \\
\verb| Meu primeiro texto| \\
\verb| \section{Minha primeira seção}| \\
\verb| \end{document}|
\end{frame}



\begin{frame}[fragile]{Sem reinventar a roda: pacotes}
O \LaTeX fornece muitos recursos pr\'e-instalados, por meio de pacotes. Esses pacotes fornecem uma variedade enorme de recursos configurados, havendo a necessidade apenas de conhecer suas caracter\'isticas (forma correta de utiliza{\c c}\~ao).
\newline
\newline
Mas como verificar a documenta{\c c}\~ao de um pacote? No linux, apenas digitar no terminal "texdoc nome-do-pacote".
\newline
\newline
Nas nossas refer\^encias est\~ao links para alguns dos muitos pacotes, dentre os mais utilizados.

\end{frame}


\begin{frame}[fragile]{Mudando o tamanho da fonte}
Pode-se alterar facilmente o tamanho da fonte utilizada nos textos. O formato padr\~ao do comando \'e \verb|\fonte{texto}|. Exemplos de formatos de fontes:\\
\begin{description}
\item[tiny]: \tiny{texto em tiny};
\item[scriptsize]: \scriptsize{texto em scriptsize};
\item[footnotesize]: \footnotesize{texto em footnotesize}; 
\item[small]: \small{texto em small};
\item[normalsize]: \normalsize{texto em normalsize};
\item[Large]: \Large{texto em Large};
\item[huge]: \huge{texto em huge}.
\end{description}
\end{frame}


\begin{frame}[fragile]{Definindo divis\~oes no texto}
Divis\~oes pr\'e-definidas em \LaTeX:\\
\verb|\part| \\
\verb|\chapter| \\
\verb|\section| \\
\verb|\subsection|\\
\verb|\subsubsection|\\
\verb|\paragraph| \\
\verb|\subparagraph|\\

\begin{itemize}
\item O estilo \textit{article} n\~ao permite o comando \verb|\chapter|;
\item A numera{\c c}\~ao de cap\'itulos/se{\c c}\~oes/subse{\c c}\~oes \'e gerada automaticamente pelo \LaTeX.
\end{itemize}
\end{frame}

%\section{Texto}
\subsection{Formatos}

\begin{frame}[fragile]{Caracteres especiais e acentua{\c c}\~ao}
\begin{itemize}
\item Os caracteres especiais \verb| $ & % # { }| podem ser obtidos pelos comandos \verb|\$ \& \% \# \{ \}|, respectivamente;
\item Acentua{\c c}\~ao pode ser obtida da seguinte maneira: \verb|\^o| produz \^o; \verb|\'e| produz \'e; \verb|\~a| produz \~a, etc;
\item O pacote \texttt{inputec} faz a convers\~ao autom\'atica dos acentos.
\end{itemize}
\end{frame}


\begin{frame}[fragile]{Hifena{\c c}\~ao}
A hifena{\c c}\~ao pode ser feita de duas formas:
\begin{enumerate}
\item por comando: \verb|\hyphenation{PYTHON com-pu-ta-dor}| (usado no pre\^ambulo);
\item no corpo do texto: \verb|com\-pu\-ta\-ção|.
\end{enumerate}
\end{frame}


\begin{frame}[fragile]{Mudando o texto}
\begin{itemize}
\item \verb|\textbf{negrito}| gera \textbf{negrito};
\item \verb|\texttt{Text}| gera \texttt{datilografado};
\item \verb|\textit{italico}| gera \textit{italico};
\item \verb|\textsc{Sans Serif}| gera \textsc{Sans Serif};
\item \verb|\emph{\Ênfase}| gera \emph{\^Enfase}.
\end{itemize}
\end{frame}


\begin{frame}[fragile]{Alinhamento}
Ambientes \emph{center}, \emph{flushleft} e \emph{flushright}:\\

\center{\verb|\center{centralizado}|}

\begin{flushleft}
\verb|\begin{flushleft}|\\
\verb|Esquerda|\\
\verb|\end{flushleft}|\\
\end{flushleft}

\begin{flushright}
\verb|\begin{flushright}|\\
\verb|Direita|\\
\verb|\end{flushright}|\\
\end{flushright}
\end{frame}


\begin{frame}[fragile]{Notas de rodap\'e}
As notas de rodap\'e s\~ao geradas pelo comando \verb|\footnote{texto}|. Exemplo:
\\
\verb|Unicentro\footnote{www.unicentro.br},| \\
\verb|Guairacá\footnote{www.faculdadeguairaca.com.br}|\\
\verb| e UTFPR\footnote{www.utfpr.edu.br}.|
\newline \newline
Produz a sa\'ida:
\newline \newline
Unicentro\footnote{www.unicentro.br}, Guairac\'a\footnote{www.faculdadeguairaca.com.br} e UTFPR\footnote{www.utfpr.edu.br}
\end{frame}

\begin{frame}[fragile]{Gerando t\'itulos de trabalhos}
\begin{itemize}
\item Declara-se:\\
\verb|\title{Meu título}|\\
\verb|\author{Meu nome}|\\
\verb|\date{Minha data} ou \date{}|
\item Omitindo-se o comando \verb|\date|, a data corrente da m\'aquina \'e utilizada;
\item Para gerar o todo t\'itulo no documento: \\
\verb|\maketitle|
\end{itemize}
\end{frame}


\begin{frame}[fragile]{Gerando t\'itulos de trabalhos - exemplo}
$\backslash$title\{Introdução ao \LaTeX\} \newline
\verb|\author{Diego Marczal e Josiel Kuk}|\newline
\verb|\date{17 de agosto de 2011}|
\end{frame}

\begin{frame}[fragile]{Listas com marcadores, numera{\c c}\~ao e descri{\c c}\~ao}
Para produzir lista com uso de marcadores, utiliza-se: \newline
\verb|\begin{itemize}|\newline
\verb|\item Primeiro item| \newline
\verb|\item Segundo item| \newline
\verb|\item Terceiro item| \newline
\verb|\end{itemize}|
\newline \newline
Produz:
\begin{itemize}
\item Primeiro item
\item Segundo item 
\item Terceiro item
\end{itemize}
\end{frame}


\begin{frame}[fragile]{Listas com marcadores, numera{\c c}\~ao e descri{\c c}\~ao}
Para produzir lista com uso de numera{\c c}\~ao, utiliza-se: \newline
\verb|\begin{enumerate}|\newline
\verb|\item Primeiro item| \newline
\verb|\item Segundo item| \newline
\verb|\item Terceiro item| \newline
\verb|\end{enumerate}|
\newline \newline
Produz:
\begin{enumerate}
\item Primeiro item
\item Segundo item 
\item Terceiro item
\end{enumerate}
\end{frame}


\begin{frame}[fragile]{Listas com marcadores, numera{\c c}\~ao e descri{\c c}\~ao}
Para produzir lista com uso de descri{\c c}\~ao, utiliza-se: \newline
\verb|\begin{description}|\newline
\verb|\item[Um] Primeiro item| \newline
\verb|\item[Dois] Segundo item| \newline
\verb|\item[Tres] Terceiro item| \newline
\verb|\end{description}|
\newline \newline
Produz:
\begin{description}
\item[Um] Primeiro item
\item[Dois] Segundo item
\item[Tres] Terceiro item
\end{description}
\end{frame}



\begin{frame}[fragile]{Listas com marcadores, numera{\c c}\~ao e descri{\c c}\~ao}
\'E poss\'ivel utilizar listas aninhadas, mesclando os tipos: \newline
\verb|\begin{itemize}|\newline
\verb|\item Primeiro item| \newline
\verb|\item Segundo item| \newline
\verb|\begin{enumerate}| \newline
\verb|\item Sub-item| \newline
\verb|\end{enumerate}| \newline
\verb|\item Terceiro item| \newline
\verb|\end{itemize}| \newline
Produz:
\begin{itemize}
\item Primeiro item 
\item Segundo item
\begin{enumerate}
\item Sub-item
\end{enumerate}
\item Terceiro item
\end{itemize}
\end{frame}


\begin{frame}[fragile]{Figuras e tabelas}
Figuras e tabelas s\~ao \emph{corpos flutuantes}, obtidos utilizando-se os ambientes \emph{figure} e \emph{table}:\newline

\verb|\begin{figure}[parâmetros]| \newline
\verb|...|\newline
\verb|\caption{texto(legenda)}| \newline
\verb|\end{figure}| \newline
e \newline
\verb|\begin{table}[parâmetros]| \newline
\verb|...|\newline
\verb|\caption{texto(legenda)}| \newline
\verb|\end{table}| \newline

\end{frame}



\begin{frame}[fragile]{Figuras e tabelas: par\^ametros}
\'E poss\'ivel especificar um ou mais par\^ametros, mas n\~ao h\'a garantia que ser\~ao seguidos:\newline
\begin{description}
\item[h] tenta posicionar o objeto na posi{\c c}\~ao em que est\'a no texto (\textbf{h}ere);
\item[t] tenta posicionar o objeto no topo da p\'agina (\textbf{t}op);
\item[b] tenta posicionar o objeto na parte inferior da p\'agina (\textbf{b}ottom);
\item[p] tenta posicionar o objeto em p\'agina especial;
\item[!] for{\c c}a o posicionamento;
\item[H] posiciona aqui ou nada feito.
\end{description}
\end{frame}

\begin{frame}[fragile]{Figuras e tabelas: exemplos}
Exemplo de tabela: \newline

\begin{table}[!ht]
\centering
\caption{Minha tabela.}
\begin{tabular}{|l|r|}     \hline
\emph{Foo}  &  \emph{Bar} \\ \hline
1000        &  2000       \\ \hline
4000        &  5000       \\ \hline
7000        &  8000       \\ \hline
\end{tabular}
\end{table}
\end{frame}

\begin{frame}[fragile]{Figuras e tabelas: exemplos}
Exemplo de figura: \newline
\begin{center}
	\includegraphics[scale=1.00]{jai.jpg}
	\captionof{figure}{IV Jornada.}
	\label{fig:jai}
\end{center}
\end{frame}

\begin{frame}[fragile]{Refer\^encias cruzadas: \emph{labels}}
O comando \verb|\label{nome referencia}| coloca uma marca naquele ponto do texto (normalmente em se{\c c}\~oes, figuras, tabelas, etc), e pode ser utilizado para se referir a ele em outra parte do texto, com o comando \verb|\ref{nome referencia}|.
\newline \newline
Exemplo: para nos referirmos a imagem do slide anterior, precisamos para referciar o seu \emph{label}, com o comando \verb|\ref{fig:jai}|. Com isso, a figura do slide anterior pode ser acessada na Figura \ref{fig:jai}.
\newline \newline
O n\'umero da p\'agina do ponto onde o  foi colocada pode ser impresso com o comando
\verb|\pageref{nome referencia}|. A a p\'agina da figura anterior \'e a \pageref{fig:jai}.
\end{frame}

%\section{Elementos Matem\'aticos}
\subsection{Elementos b\'asicos}

\begin{frame}[fragile]{Elementos b\'asicos}
\begin{itemize}
\item Utiliza-se \verb|$...$| para produzir f\'ormulas dentro de um par\'agrafo;
\item Utiliza-se \verb|\[...]| para produzir equa{\c c}\~oes destacadas do par\'agrafo;
\item Utiliza-se \verb|\begin{equation}...\end{equation}| para poder referenciar a equa{\c c}\~ao usando \verb|\ref{}|.
\end{itemize}
\end{frame}


\begin{frame}[fragile]{Exemplos}

\verb|A Equação \ref{eqn:exemplo} é apresentada abaixo:|\newline
\verb|\newline| \newline
\verb|\begin{equation}\label{eqn:exemplo}|\newline
\verb|2x^2-3x+1=0|\newline
\verb|\end{equation}|\newline

Isso produz:\newline

A Equação \ref{eqn:exemplo} é apresentada abaixo: \newline
\begin{equation}\label{eqn:exemplo}
2x^2-3x+1=0
\end{equation}
\end{frame}


\begin{frame}[fragile]{Exemplos}

\verb|A Equação \ref{eqn:exemplo2} é apresentada abaixo:|\newline
\verb|\newline| \newline
\verb|\begin{equation}\label{eqn:exemplo2}|\newline
\verb|\frac{-b\pm\sqrt{b^2-4ac}}{2a}|\newline
\verb|\end{equation}|\newline

Isso produz:\newline

A Equação \ref{eqn:exemplo2} é apresentada abaixo: \newline
\begin{equation}\label{eqn:exemplo2}
x=\frac{-b\pm\sqrt{b^2-4ac}}{2a}
\end{equation}
\end{frame}



\begin{frame}[fragile]{Exemplos}
\begin{itemize}
\item \verb|$\sqrt[3]{8}=2$| produz $\sqrt[3]{8}=2$.
\item \verb|$a_n, x_i^2, x^{2n}$| produz $a_n, x_i^2, x^{2n}$.
\item \verb|$\int\limits_a^b f(x)dx$| produz $\int\limits_a^b f(x)dx$.
\item \verb|$\sum_{i=1}^n a_i$| produz $\sum_{i=1}^n a_i$.
\item \verb|$\sum\limits_{i=1}^n a_i$| produz $\sum\limits_{i=1}^n a_i$.
\item \verb|${n+1\choose k}={n\choose k}+{n\choose k-1}$| produz ${n+1\choose k}={n\choose k}+{n\choose k-1}$.
\end{itemize} 
\end{frame}



\begin{frame}[fragile]{Fun{\c c}\~oes Matem\'aticas}

Algumas das fun{\c c}\~oes pr\'e-definidas:
\newline \newline
\verb|\arccos \arcsin \arctan \arg \cos|
\verb|\cosh \cot \coth \csc \deg \det|
\verb|\dim \exp \gcd \hom \inf \ker \lg|
\verb|\lim \liminf \limsup \ln \log \max|
\verb|\min \Pr \sec \sin \sinh \sup \tan \tanh|

\end{frame}


\begin{frame}[fragile]{Matrizes}
Permite descrever tabelas e matrizes. Exemplo:
\begin{verbatim}
\begin{array}{clcr}
a+b+c & uv     & x-y & 27 \\
a+b   & u+v    & z   & 134 \\
a     & 3u+vw  & xyz & 2.978 \\ 
\end{array}
\end{verbatim}

Produz:

$\begin{array}{clcr}
$a+b+c$ & uv     & $x-y$ & 27 \\
$a+b$   & $u+v$    & z   & 134 \\
a     & $3u+vw$  & xyz & 2.978 \\

\end{array}$

\end{frame}


\begin{frame}[fragile]{Matrizes}
Matrizes podem ser obtidas usando-se delimitadores – \{, $[$, $($. Para indicar se o delimitador \'e o esquerdo ou o direito, deve-se anteceder o delimitador por \verb|\left| ou \verb|\right|. Exemplo:

\verb|\[ \left [| \newline
\verb|\begin{array}{clcr}| \newline
\verb|$a+b+c$ & uv     & $x-y$ & 27 \\| \newline
\verb|$a+b$   & $u+v$    & z   & 134 \\| \newline
\verb|a     & $3u+vw$  & xyz & 2.978 \\| \newline
\verb|\end{array}| \newline
\verb|\right ] \]| \newline

\[ \left [|
\begin{array}{clcr}
$a+b+c$ & uv     & $x-y$ & 27 \\
$a+b$   & $u+v$    & z   & 134 \\
a     & $3u+vw$  & xyz & 2.978 \\
\end{array}
\right ] \]

\end{frame}


\begin{frame}[fragile]{Exemplo}

\verb|\[ \left (| \newline
\verb|\begin{array}{ccc}| \newline
\verb|a_{11} & \cdots & a_{1n} \\| \newline
\verb|\vdots & \ddots & \vdots \\ | \newline
\verb|a_{m1} & \cdots & a_{mn}| \newline
\verb|\end{array}| \newline
\verb|\right ) \] | \newline

\[ \left (
\begin{array}{ccc}
a_{11} & \cdots & a_{1n} \\
\vdots & \ddots & \vdots \\
a_{m1} & \cdots & a_{mn}
\end{array} \right ) \]


\end{frame}

\section{Refer\^encias Bibliogr\'aficas}
\subsection{Elementos b\'asicos}

\begin{frame}[fragile]{Elementos b\'asicos}
\begin{itemize}
\item BiB\TeX \'e um programa externo que permite definir refer\^encias bibliogr\'aficas;
\item Usa uma rela{\c c}\~ao de refer\^encias definidas em um arquivo \texttt{.bib};
\item S\~ao importadas somente as refer\^encias indicadas pelos comandos \verb|\cite| e \verb|\nocite|;
\item O programa \texttt{bibtex} l\^e o arquivo \texttt{.aux} gerado pelo \LaTeX;
\item O comando \verb|\bibliography{nome}| informa que a bibliografia encontra-se no arquivo \texttt{nome.bib};
\item O comando \verb|\bibliographystyle{...}| define o estilo da bibliografia a ser produzida (existem v\'arios estilos).
\end{itemize}
\end{frame}

\subsection{BiB\TeX}

\begin{frame}[fragile]{BiB\TeX}
Estrutura do arquivo .bib: cont\'em uma sequ\^encia de entradas, sendo cada entrada definida como:\newline
\newline 
\verb|@tipo{rótulo, chave={valor}, chave={valor},...}|
\newline \newline
Tipos de entradas mais comuns:
\newline
\begin{description}
\item[book] livro;
\item[inproceedings] artigo em anais de evento;
\item[article] artigo em peri\'odico.
\end{description}
\end{frame}

\begin{frame}[fragile,allowframebreaks]{BiB\TeX: Exemplo}

\verb|@Book{livropca,| \newline
\verb|author = {Ian T. Jolliffe},| \newline
\verb|publisher = {Springer-Verlag},| \newline
\verb|title = {Principal Component Analysis},| \newline
\verb|year = {2002},| \newline
\verb|note = {ISBN 0387954422}| \newline
\verb|}| \newline

\cite{latex_wikibooks}

\end{frame} 

%\section{Refer\^encias}

\begin{frame}
\begin{block}{Refer\^encias}
\begin{itemize}
\item \LaTeX Project (http://www.latex-project.org/);
\item \LaTeX Wikibook (http://en.wikibooks.org/wiki/LaTeX/);
\item The Not So Short Introduction to LaTeX (http://www.ctan.org/tex-archive/info/lshort/)
\end{itemize}
\end{block}
\end{frame}



\begin{frame}[allowframebreaks]{Bibliography}
\bibliographystyle{plain}
\bibliography{apresentacao}
\end{frame}
%\bibliography{apresentacao}
%\bibliographystyle{plain}


\end{document}



