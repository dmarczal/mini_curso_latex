\section{Texto}
\subsection{Formatos}

\begin{frame}[fragile]{Caracteres especiais e acentua{\c c}\~ao}
\begin{itemize}
\item Os caracteres especiais \verb| $ & % # { }| podem ser obtidos pelos comandos \verb|\$ \& \% \# \{ \}|, respectivamente;
\item Acentua{\c c}\~ao pode ser obtida da seguinte maneira: \verb|\^o| produz \^o; \verb|\'e| produz \'e; \verb|\~a| produz \~a, etc;
\item O pacote \texttt{inputec} faz a convers\~ao autom\'atica dos acentos.
\end{itemize}
\end{frame}


\begin{frame}[fragile]{Hifena{\c c}\~ao}
A hifena{\c c}\~ao pode ser feita de duas formas:
\begin{enumerate}
\item por comando: \verb|\hyphenation{PYTHON com-pu-ta-dor}| (usado no pre\^ambulo);
\item no corpo do texto: \verb|com\-pu\-ta\-ção|.
\end{enumerate}
\end{frame}


\begin{frame}[fragile]{Mudando o texto}
\begin{itemize}
\item \verb|\textbf{negrito}| gera \textbf{negrito};
\item \verb|\texttt{Text}| gera \texttt{datilografado};
\item \verb|\textit{italico}| gera \textit{italico};
\item \verb|\textsc{Sans Serif}| gera \textsc{Sans Serif};
\item \verb|\emph{\Ênfase}| gera \emph{\^Enfase}.
\end{itemize}
\end{frame}


\begin{frame}[fragile]{Alinhamento}
Ambientes \emph{center}, \emph{flushleft} e \emph{flushright}:\\

\center{\verb|\center{centralizado}|}

\begin{flushleft}
\verb|\begin{flushleft}|\\
\verb|Esquerda|\\
\verb|\end{flushleft}|\\
\end{flushleft}

\begin{flushright}
\verb|\begin{flushright}|\\
\verb|Direita|\\
\verb|\end{flushright}|\\
\end{flushright}
\end{frame}


\begin{frame}[fragile]{Notas de rodap\'e}
As notas de rodap\'e s\~ao geradas pelo comando \verb|\footnote{texto}|. Exemplo:
\\
\verb|Unicentro\footnote{www.unicentro.br},| \\
\verb|Guairacá\footnote{www.faculdadeguairaca.com.br}|\\
\verb| e UTFPR\footnote{www.utfpr.edu.br}.|
\newline \newline
Produz a sa\'ida:
\newline \newline
Unicentro\footnote{www.unicentro.br}, Guairac\'a\footnote{www.faculdadeguairaca.com.br} e UTFPR\footnote{www.utfpr.edu.br}
\end{frame}

\begin{frame}[fragile]{Gerando t\'itulos de trabalhos}
\begin{itemize}
\item Declara-se:\\
\verb|\title{Meu título}|\\
\verb|\author{Meu nome}|\\
\verb|\date{Minha data} ou \date{}|
\item Omitindo-se o comando \verb|\date|, a data corrente da m\'aquina \'e utilizada;
\item Para gerar o todo t\'itulo no documento: \\
\verb|\maketitle|
\end{itemize}
\end{frame}


\begin{frame}[fragile]{Gerando t\'itulos de trabalhos - exemplo}
$\backslash$title\{Introdução ao \LaTeX\} \newline
\verb|\author{Diego Marczal e Josiel Kuk}|\newline
\verb|\date{17 de agosto de 2011}|
\end{frame}

\begin{frame}[fragile]{Listas com marcadores, numera{\c c}\~ao e descri{\c c}\~ao}
Para produzir lista com uso de marcadores, utiliza-se: \newline
\verb|\begin{itemize}|\newline
\verb|\item Primeiro item| \newline
\verb|\item Segundo item| \newline
\verb|\item Terceiro item| \newline
\verb|\end{itemize}|
\newline \newline
Produz:
\begin{itemize}
\item Primeiro item
\item Segundo item 
\item Terceiro item
\end{itemize}
\end{frame}


\begin{frame}[fragile]{Listas com marcadores, numera{\c c}\~ao e descri{\c c}\~ao}
Para produzir lista com uso de numera{\c c}\~ao, utiliza-se: \newline
\verb|\begin{enumerate}|\newline
\verb|\item Primeiro item| \newline
\verb|\item Segundo item| \newline
\verb|\item Terceiro item| \newline
\verb|\end{enumerate}|
\newline \newline
Produz:
\begin{enumerate}
\item Primeiro item
\item Segundo item 
\item Terceiro item
\end{enumerate}
\end{frame}


\begin{frame}[fragile]{Listas com marcadores, numera{\c c}\~ao e descri{\c c}\~ao}
Para produzir lista com uso de descri{\c c}\~ao, utiliza-se: \newline
\verb|\begin{description}|\newline
\verb|\item[Um] Primeiro item| \newline
\verb|\item[Dois] Segundo item| \newline
\verb|\item[Tres] Terceiro item| \newline
\verb|\end{description}|
\newline \newline
Produz:
\begin{description}
\item[Um] Primeiro item
\item[Dois] Segundo item
\item[Tres] Terceiro item
\end{description}
\end{frame}



\begin{frame}[fragile]{Listas com marcadores, numera{\c c}\~ao e descri{\c c}\~ao}
\'E poss\'ivel utilizar listas aninhadas, mesclando os tipos: \newline
\verb|\begin{itemize}|\newline
\verb|\item Primeiro item| \newline
\verb|\item Segundo item| \newline
\verb|\begin{enumerate}| \newline
\verb|\item Sub-item| \newline
\verb|\end{enumerate}| \newline
\verb|\item Terceiro item| \newline
\verb|\end{itemize}| \newline
Produz:
\begin{itemize}
\item Primeiro item 
\item Segundo item
\begin{enumerate}
\item Sub-item
\end{enumerate}
\item Terceiro item
\end{itemize}
\end{frame}


\begin{frame}[fragile]{Figuras e tabelas}
Figuras e tabelas s\~ao \emph{corpos flutuantes}, obtidos utilizando-se os ambientes \emph{figure} e \emph{table}:\newline

\verb|\begin{figure}[parâmetros]| \newline
\verb|...|\newline
\verb|\caption{texto(legenda)}| \newline
\verb|\end{figure}| \newline
e \newline
\verb|\begin{table}[parâmetros]| \newline
\verb|...|\newline
\verb|\caption{texto(legenda)}| \newline
\verb|\end{table}| \newline

\end{frame}



\begin{frame}[fragile]{Figuras e tabelas: par\^ametros}
\'E poss\'ivel especificar um ou mais par\^ametros, mas n\~ao h\'a garantia que ser\~ao seguidos:\newline
\begin{description}
\item[h] tenta posicionar o objeto na posi{\c c}\~ao em que est\'a no texto (\textbf{h}ere);
\item[t] tenta posicionar o objeto no topo da p\'agina (\textbf{t}op);
\item[b] tenta posicionar o objeto na parte inferior da p\'agina (\textbf{b}ottom);
\item[p] tenta posicionar o objeto em p\'agina especial;
\item[!] for{\c c}a o posicionamento;
\item[H] posiciona aqui ou nada feito.
\end{description}
\end{frame}

\begin{frame}[fragile]{Figuras e tabelas: exemplos}
Exemplo de tabela: \newline

\begin{table}[!ht]
\centering
\caption{Minha tabela.}
\begin{tabular}{|l|r|}     \hline
\emph{Foo}  &  \emph{Bar} \\ \hline
1000        &  2000       \\ \hline
4000        &  5000       \\ \hline
7000        &  8000       \\ \hline
\end{tabular}
\end{table}
\end{frame}

\begin{frame}[fragile]{Figuras e tabelas: exemplos}
Exemplo de figura: \newline
\begin{center}
	\includegraphics[scale=1.00]{jai.jpg}
	\captionof{figure}{IV Jornada.}
	\label{fig:jai}
\end{center}
\end{frame}

\begin{frame}[fragile]{Refer\^encias cruzadas: \emph{labels}}
O comando \verb|\label{nome referencia}| coloca uma marca naquele ponto do texto (normalmente em se{\c c}\~oes, figuras, tabelas, etc), e pode ser utilizado para se referir a ele em outra parte do texto, com o comando \verb|\ref{nome referencia}|.
\newline \newline
Exemplo: para nos referirmos a imagem do slide anterior, precisamos para referciar o seu \emph{label}, com o comando \verb|\ref{fig:jai}|. Com isso, a figura do slide anterior pode ser acessada na Figura \ref{fig:jai}.
\newline \newline
O n\'umero da p\'agina do ponto onde o  foi colocada pode ser impresso com o comando
\verb|\pageref{nome referencia}|. A a p\'agina da figura anterior \'e a \pageref{fig:jai}.
\end{frame}
