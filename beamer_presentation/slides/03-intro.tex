\section{Intro}

\subsection{O que é o Beamer}
%----------------------------------------------------------------------------%

\begin{frame}

  \begin{itemize}[<+->]
     \item Beamer é uma classe do \LaTeX para criar apresentações.
     \item Preparar apresentações com o Beamer é muito diferente do que utilizando
           os editores WYSIWYG, como o OpenOffice Impress, Apple Keynote, PowerPoint, etc..
     \item Uma apresentação Beamer é como qualquer outro documento do \LaTeX.
     \item Por isso, para usar o Beamer é necessário conhecer o \LaTeX.
  \end{itemize}

\end{frame}

%----------------------------------------------------------------------------%

\subsection{História}

\begin{frame}
   \begin{itemize}[<+->]
      \item Criado por \textbf{Till Tantau}.
      \item Till criou o Beamer para fazer a apresentação da Tese do seu Doutorado em 2003.
      \item Em 2010, \textbf{Joseph Wright e Vedran Miletić} passaram a manter o Beamer.
   \end{itemize}
\end{frame}

%----------------------------------------------------------------------------%
\section{Vantagens}
\subsection{Por que Utilizar Beamer?}

\begin{frame}
  \begin{itemize}[<+->]
     \item Os comandos padrões do \LaTeX também funcionam no Beamer.
     \item O sumário e as formações da apresentação são automaticamente criadas. Incluindo
          links para seção e subseção.
     \item É fácil criar animações e efeitos.
     \item Possui temas que permitem mudar aparência da sua apresentação com apenas um comando.
     \item Sendo cada tema desenvolvido para ser reutilizável em qualquer apresentação Beamer.
  \end{itemize}

\end{frame}

%----------------------------------------------------------------------------%

\begin{frame}
   \begin{itemize}[<+->]
      \item Os layouts, cores e as fontes podem ser facilmente alteradas tanto globalmente como
            localmente.
      \item É possível criar apresentações usando o mesmo código \LaTeX escrito para artigos.
      \item O saída gerada pela compilação dos códigos é geralmente no formato pdf.
      \item \alert{Sua apresentação terá \textbf{exatamente} mesmo formato em qualquer computador e SO!}
   \end{itemize}
\end{frame}

%----------------------------------------------------------------------------%

%Links
%  Página Inicial
%  Apoio (rails mg)
%  Sobre o evento
%  Programação
%  Palestrantes
%  Fale conosco
%     form
%     telefone
%     email
% Cadastro de emails
% Twitter
% FaceBook
% Orkut
% www.soeaa.com.br
