\section{Templates}
\subsection{Como e por que Templates}

\begin{frame}
  \begin{itemize}[<+->]
     \item  O jeito mais rápido de começar com o Beamer é através do uso de templates.
     \item  Você pode gerar sua template através de scripts.
     \item  Vamos utilizar um \href{http:\\www.inf.ufpr.br/diego/minicurso\_latex}{\textcolor{blue}{script}}.
     \item  Crie uma basta \textbf{scripts} no seu home.
     \item  Descompacte os arquivos na pasta criada.
     \item  Execute o seguinte comando \textbf{./install}
  \end{itemize}
\end{frame}

\begin{frame}
  \frametitle{Teste o template}
  \begin{itemize}[<+->]
     \item Para criar uma apresentação execute o comando \textbf{latex\_new} nome da apresentação.
     \item Não use acentos nem espaços para a nome da apresentação.
     \item Para compilar entre no diretório do projeto e execute o comando \textbf{make}.
     \item Conhecendo a estrutura de diretórios e o arquivo \textbf{Makefile}.
  \end{itemize}
\end{frame}

\begin{frame}[fragile]
  \frametitle{Definindo informações}

    \only<1-5>{As primeiras alterações são para as informações chaves da sua apresentação}.

    \begin{block}{Alterações necessárias}<2->
      \begin{itemize}
         \item <2-> \mint{tex} |\title[Titulo menor]{Titulo maior}|
         \item <3-> \mint{tex} |\subtitle[Titulo menor]{Subtitulo maior}|
         \item <4-> \mint{tex} |\author[sobrenome]{nome}|
         \item <5-> \mint{tex} |\date[18/08/2011]{18 de agosto de 2011}|
     \end{itemize}
    \end{block}
\end{frame}
