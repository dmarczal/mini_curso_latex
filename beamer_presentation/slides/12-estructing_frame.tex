\section{Estrutura}
\subsection{Estruturando um frame}
%------------------------------------------------------------%

\begin{frame}
  \frametitle{Estruturando um frame}

  Beamer oferece diversas formas de estruturar um frame, dentre elas veremos:

  \begin{itemize}
     \item Columns
     \item Blocks
     %\item Boxes (Borders)
  \end{itemize}

\end{frame}

%------------------------------------------------------------%

\begin{frame}[fragile]
  \frametitle{Colunas}

  \begin{block}<1->{Código para uma coluna de um Coluna}
    \begin{minted}[fontsize=\scriptsize]{tex}
        \begin{columns}
          \column[.x\textwidth]
          ...
          \column[.x\textwidth]
          ...
          \column[.x\textwidth]
          ...
        \end{columns}
    \end{minted}
  \end{block}

  \begin{block}<2->{}
    \begin{itemize}
       \item O beamer dispõe de um ambiente muito útil para dividir o
            slide, ou partes dele, em múltiplas colunas. \textcolor{red}{ $\backslash$ begin\{columns\}}
       \item Ele funciona como o ambiente itemize, para começar uma nova coluna usa-se o comando
             \textcolor{red} { $\backslash$ column[largura]}
       \item Onde \textcolor{red}{x} é a porcentagem da largura do slide.
    \end{itemize}

  \end{block}

\end{frame}

%------------------------------------------------------------%

\begin{frame}[fragile]
  \frametitle{Exemplo:}

  \begin{minted}[fontsize=\scriptsize]{tex}
    \begin{columns}
      \column{.3\textwidth}
        Coluna 1
      \column{.3\textwidth}
        Coluna 2
    \end{columns}
  \end{minted}

  \begin{block}{Resultado}
    \begin{center}
     \begin{columns}
       \column{.3\textwidth}
         Coluna 1
      \column{.3\textwidth}
        Coluna 2
    \end{columns}
    \end{center}
  \end{block}
\end{frame}
\begin{frame}[fragile]
  \frametitle{Colunas}

  \begin{block}{Código para uma coluna de um Coluna}
    \begin{minted}[fontsize=\scriptsize]{tex}
        \begin{columns}
          \column[.x\textwidth]
          ...
          \column[.x\textwidth]
          ...
          \column[.x\textwidth]
          ...
        \end{columns}
    \end{minted}
  \end{block}

  \begin{block}{}
    \begin{itemize}
       \item O beamer dispõe de um ambiente muito útil para dividir o
            slide, ou partes dele, em múltiplas colunas.  \textcolor{red}{ $\backslash$begin\{columns\}}
       \item Ele funciona como o ambiente itemize, para começar uma nova coluna usa-se o comando
             \textcolor{red} { $\backslash$column[largura]}
       \item Onde \textcolor{red}{x} é a porcentagem da largura do slide.
    \end{itemize}

  \end{block}

\end{frame}

%------------------------------------------------------------%

\begin{frame}[fragile]
  \frametitle{Exemplo:}

  \begin{minted}[fontsize=\scriptsize]{tex}
    \begin{columns}
      \column{.3\textwidth}
        Coluna 1
      \column{.3\textwidth}
        Coluna 2
    \end{columns}
  \end{minted}

  \begin{block}{Resultado}
    \begin{center}
     \begin{columns}
       \column{.3\textwidth}
         Coluna 1
      \column{.3\textwidth}
        Coluna 2
    \end{columns}
    \end{center}
  \end{block}
\end{frame}

%------------------------------------------------------------%

\begin{frame}[fragile]
  \frametitle{Blocos}

  \begin{minted}[fontsize=\scriptsize]{tex}
    \begin{block}{Latex com Beamer}
      Apresentando o Beamer....
    \end{block}
  \end{minted}

   \begin{block}{Latex com Beamer}
      Apresentando o Beamer....
   \end{block}

\end{frame}

%------------------------------------------------------------%

\begin{frame}[fragile]
  \frametitle{Blocos em duas colunas}

  \begin{minted}[fontsize=\scriptsize]{tex}

    \begin{columns}
      \begin{column}[l]{5cm}
         \begin{block}{Latex com Beamer}
            Apresentando o Beamer.... 1
         \end{block}
      \end{column}

      \begin{column}[r]{5cm}
         \begin{block}{Latex com Beamer}
            Apresentando o Beamer.... 2
         \end{block}
      \end{column}
    \end{columns}
  \end{minted}

  \begin{columns}
    \begin{column}[l]{5cm}
       \begin{block}{Latex com Beamer}
          Apresentando o Beamer.... 1
       \end{block}
    \end{column}

    \begin{column}[r]{5cm}
       \begin{block}{Latex com Beamer}
          Apresentando o Beamer.... 2
       \end{block}
    \end{column}
  \end{columns}

\end{frame}

%------------------------------------------------------------%

%\begin{frame}[fragile]
%  \frametitle{Caixa de textos}
%
%  \begin{block}{Caixas de textos}
%
%    Para usar as caixas de textos é necessário incluir o pacote \textbf{fancybox} através do comando:
%    $\backslash$usepackage\{fancybox\}
%
%    \begin{columns}
%      \begin{column}[l]{5cm}
%        \begin{minted}[fontsize=\scriptsize]{tex}
%          \shadowbox{Shadow Box}
%          \fbox{fbox}
%          \doublebox{Double Box}
%          \ovalbox{oval Box}
%          \Ovalbox{Oval Box}
%        \end{minted}
%      \end{column}
%
%      \begin{column}[r]{5cm}
%          \shadowbox{Shadow Box}
%          \fbox{fbox}
%          \doublebox{Double Box}
%          \ovalbox{oval Box}
%          \Ovalbox{Oval Box}
%      \end{column}
%
%    \end{columns}
%  \end{block}
%\end{frame}

